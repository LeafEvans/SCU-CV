\documentclass[11pt]{article}
\usepackage{xltxtra}
\usepackage{bookmark}
\usepackage{hyperref}
\hypersetup{hidelinks}
\usepackage{url}
\urlstyle{tt}
\usepackage{multicol}
\usepackage{xcolor}
\usepackage{calc}
\usepackage{graphicx}
\usepackage{tikz}
\usetikzlibrary{calc}
\usepackage{fontspec}
\usepackage{xeCJK}
\usepackage{relsize}
\usepackage{xspace}
\usepackage{fontawesome5}
\usepackage{titlesec}
\usepackage{enumitem}
\usepackage{siunitx}
\usepackage{amssymb}
\usepackage{tabularx}
\usepackage{fancybox}
\usepackage{float}

% 设置:取消中文字符与数字之间的间隔
\CJKsetecglue{}

% C++的美观写法
\protected\def\Cpp{{C\nolinebreak[4]\hspace{-.05em}\raisebox{.28ex}{\relsize{-1}++}}\xspace}

% 全局段落缩进取消
\setlength{\parindent}{0pt}

% 取消页码
\pagenumbering{gobble}

% 调整itemize列表的顶部间距和左边距
\setlist[itemize]{topsep=0em, leftmargin=*}
% 调整enumerate列表的顶部间距和左边距
\setlist[enumerate]{topsep=0em, leftmargin=*}

% 一级标题格式设置
\titleformat{\section}
  {\LARGE\bfseries\raggedright}  % 字体改为LARGE,粗体,左对齐
  {}{0em}                       % 不使用数字
  {}                            % 可添加代码
  [{\color{SCU_Grey}\titlerule}] % 标题下方加一条灰色线
\titlespacing*{\section}{0cm}{*1.2}{*1.2}

% 二级标题格式设置
\titleformat{\subsection}
  {\large\bfseries\raggedright} % 字体改为large,粗体,左对齐
  {}{0em}                       % 不使用数字
  {}                            % 可添加代码
  []
\titlespacing*{\subsection}{0cm}{*1.2}{*1.2}

% 页边距设置
\usepackage[
  a4paper,
  left=1.2cm,
  right=1.2cm,
  top=1.5cm,
  bottom=1cm,
  nohead
]{geometry}

% 表格行间距设置为1.2倍
\renewcommand{\arraystretch}{1.2}
% 正文行间距设置为1.2倍
\linespread{1.2}
% 中文字符间距设置
\renewcommand{\CJKglue}{\hskip 0.05em}

% 自定义字体设置
% 英文字体:NotoSerifSC
\setmainfont[
  Path=fonts/,
  Extension=.otf,
  BoldFont=*-Bold,
]{NotoSerifSC}
% 中文字体:NotoSerifSC
\setCJKmainfont[
  Path=fonts/,
  Extension=.otf,
  BoldFont=*-Bold,
]{NotoSerifSC}

% 自定义配色:四川大学标识色
\definecolor{SCU_Red}{RGB}{180, 16, 10}
\definecolor{SCU_Grey}{RGB}{135, 161, 161}


% 填写个人信息
% 学院信息
\newcommand{\school}{计算机学院 | College of Computer Science} 

% 联系方式
\newcommand{\contact}{
    \scriptsize
    \textcolor{white}{
        % 邮箱
        \faEnvelope \quad \href{mailto:youremail@scu.edu.com}{youremail@scu.edu.com}
        \hspace{4em}
        % 手机号
        \faPhone \quad  130-6666-0000
        % GitHub
        \hspace{4em}
        \faGithub \quad \href{https://github.com/xxxx}{https://github.com/xxxx}
    }
}

%%%%%%%%%%%%%%%%%%%%
% 简历正文
%%%%%%%%%%%%%%%%%%%%
\begin{document}

% 页眉:校名
\begin{tikzpicture}[remember picture, overlay]
    \node[anchor=north, inner sep=0pt](header) at (current page.north){
        \includegraphics[width=\paperwidth]{images/header.png}
    };
    \node[anchor=west](school_logo) at (header.west){
        \hspace{0.5cm}
        \includegraphics[width=0.15\textwidth]{images/logo_1.png}
    };
    \node[anchor=east](school_name) at(header.east){
        \textcolor{white}{\textbf{\school}}
        \hspace{0.5cm}
    };
\end{tikzpicture}
\vspace{-3.5em}

% 页脚:联系方式
\begin{tikzpicture}[remember picture, overlay]
    \node[anchor=south, inner sep=0pt](footer) at (current page.south){
        \includegraphics[width=\paperwidth]{images/footer.png}
    };
    \node[anchor=center] at(footer.center){\contact};
\end{tikzpicture}

% 背景
\begin{tikzpicture}[remember picture, overlay]
    \node[opacity=0.05] at(current page.center){
        \includegraphics[width=0.7\paperwidth, keepaspectratio]{images/logo_2.png}
    };
\end{tikzpicture}

% 个人信息部分
\begin{figure}[h]
    \begin{minipage}{0.82\textwidth}
        \section{\makebox[\widthof{\faAddressCard}][c]{\color{SCU_Red}{\faAddressCard}}\quad 个人信息}
        \begin{tabularx}{\linewidth}{p{\widthof{出生日期:}}Xp{\widthof{政治面貌:}}X}
            姓\ \ \ \ \ \ \ \ 名: & 丁真珍珠 & 
            性\ \ \ \ \ \ \ \ 别: & 男  \\
            出生年月: & 2001年5月7日 & 
            政治面貌: & 群众 \\
        \end{tabularx}
    \end{minipage}
    \hspace{2em}
    % 右边照片部分
    \begin{minipage}{0.12\textwidth}
        \setlength{\fboxsep}{0pt}
        \doublebox{\includegraphics[width=\linewidth]{images/avatar.jpeg}}
    \end{minipage}
\end{figure}
\vspace{-1em}

% 教育背景
\section{\makebox[\widthof{\faGraduationCap}][c]{\color{SCU_Red}{\faGraduationCap}}\quad 教育背景}
% 本科
{\large \textbf{四川大学锦江学院}},专科 \hfill {四川,成都} \\
\href{https://www.scu.edu.cn/}{你的学院},专业:你的专业 \hfill {2000年9月-2010年6月} \\
主修课程:课程1、课程2、课程3\ 等。

\vspace{0.5em}
% 硕士
{\large \textbf{四川大学}},本科 \hfill {四川,成都} \\
{{吴玉章书院}},专业:你的专业 \hfill {2010年9月-2020年6月} \\
主修课程:课程1、课程2、课程3\ 等。

\vspace{0.5em}
{\large \textbf{四川大学}},硕士 \hfill {四川,成都} \\
计算机学院,导师:\href{导师的个人主页.site}{导师名字} \hfill {2020年9月-至今} \\
研究方向:方向1、方向2\ 等。

% 科研成果
\section{\makebox[\widthof{\faGraduationCap}][c]{\color{SCU_Red}{\faGraduationCap}}\quad 科研成果}
This is One of Your Paper Published in Conference A. \\
\textbf{Tenzin Tsundue}, Alice Leaf. \hfill 发表于 \textbf{Conference A}(CCF-A类会议) 

\vspace{0.5em}
This is Another Paper. \\
\textbf{Tenzin Tsundue}, Alice Leaf, Evans Leaf. \hfill 发表于 \textbf{Conference B} (CCF-A类会议)

\vspace{0.5em}
This is A Journal Paper. \\
\textbf{Tenzin Tsundue}, Alice Leaf, Evans Leaf. \hfill 发表于 \textbf{Journal C} (SCI-1区)

% 项目与教学经历
\section{\makebox[\widthof{\faChalkboardTeacher}][c]{\color{SCU_Red}{\faChalkboardTeacher}}\quad 项目与教学}
\vspace{0.5em}
{\large{\textbf{项目名称}}} \hfill {横向/纵向项目-已完结/进行中} \\
你在项目中扮演的角色 \hfill 2020年9月至2021年9月 \\
项目简介。

\vspace{1em}
{\large{\textbf{某某主题讨论班}}},主讲/参与 \hfill {2020年夏季} \\
主要内容:内容1,内容2\ 等。

\vspace{1em}
{\large{\textbf{课程名称}}},助教 \hfill {2021年夏季} \\
主要内容:内容1,内容2\ 等。

% 技能
\section{\makebox[\widthof{\faWrench}][c]{\color{SCU_Red}{\faWrench}}\quad 技能}
\vspace{0.5em}
\begin{itemize}
    \item 英语:六级800分、托福200分;
    \item 编程:Python, C++, MATLAB, C.
\end{itemize}

\newpage

% 页眉:校名+学院名(新页)
\begin{tikzpicture}[remember picture, overlay]
    \node[anchor=north, inner sep=0pt](header) at (current page.north){
        \includegraphics[width=\paperwidth]{images/header.png}
    };
    \node[anchor=west](school_logo) at (header.west){
        \hspace{0.5cm}
        \includegraphics[width=0.15\textwidth]{images/logo_1.png}
    };
    \node[anchor=east](school_name) at(header.east){
        \textcolor{white}{\textbf{\school}}
        \hspace{0.5cm}
    };
\end{tikzpicture}
\vspace{-4em}

% 页脚:联系方式
\begin{tikzpicture}[remember picture, overlay]
    \node[anchor=south, inner sep=0pt](footer) at (current page.south){
        \includegraphics[width=\paperwidth]{images/footer.png}
    };
    \node[anchor=center] at(footer.center){\contact};
\end{tikzpicture}

% 背景
\begin{tikzpicture}[remember picture, overlay]
    \node[opacity=0.1] at(current page.center){
        \includegraphics[width=0.7\paperwidth, keepaspectratio]{images/logo_2.png}
    };
\end{tikzpicture}

% 竞赛经历
\section{\makebox[\widthof{\faTrophy}][c]{\color{SCU_Red}{\faTrophy}}\quad 竞赛经历}
\vspace{-1em}
\begin{table}[h!]
    \begin{tabularx}{\textwidth}{Xp{\widthof{第零负责人}}p{\widthof{国家级-第100名}}p{\widthof{2030年13月}}}
        \textbf{比赛1} & 第一负责人 & 国家级-第10名 & 2023年4月 \\
        \textbf{比赛2} & 个人参赛 & 国家级-一等奖 & 2023年8月 \\
        \textbf{比赛3} & 个人参赛 & 省级-一等奖 & 2022年12月 \\
    \end{tabularx}
\end{table}

% 技能特长
\section{\makebox[\widthof{\faWrench}][c]{\color{SCU_Red}{\faWrench}}\quad 技能特长}
\vspace{0.5em}
\begin{itemize}
    \item 熟练使用\Cpp 、Python、Matlab编程语言。
    \item 熟悉Windows与Linux端开发。
    \item 熟练使用Tensorflow,Pytorch等深度学习框架。
    \item 熟练掌握\Cpp 与Python环境下OpenCV与Qt应用的开发。
\end{itemize}

% 所获荣誉
\section{\makebox[\widthof{\faStar}][c]{\color{SCU_Red}{\faStar}}\quad 所获荣誉}
\vspace{-1em}
\begin{multicols}{2}
    \begin{itemize}
        \item 2021年 四川文化旅游宣传推广大使
        \item 2021年 四川生态环境保护大使
        \item 2020年 理塘县旅游大使
        \item 2022年 《天天向上》综艺节目成员
        \item 2023年 央视纪录片《跟着丁真探乡村》参与者
        \item 2024年 《岛屿少年》综艺节目参与者
    \end{itemize}
\end{multicols}


% 其他
\section{\makebox[\widthof{\faInfo}][c]{\color{SCU_Red}{\faInfo}}\quad 其他}
\begin{itemize}
    \item 英语水平-CET6级xxx分
    \item 计算机几级证书
    \item 教师资格证:xxx
\end{itemize}

\end{document}
